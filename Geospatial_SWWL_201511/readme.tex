\documentclass[onecolumn,aps,superscriptaddress,preprint]{revtex4}
\usepackage{amsfonts}
\usepackage{epsfig}
\usepackage{amsmath,amssymb}
\usepackage{times}
\usepackage{xcolor}
\renewcommand{\thefigure}{S\arabic{figure}}
\newcommand{\grad}{\ensuremath{^{\circ}}}

\begin{document}

\title{Description for Source Codes and Data}

\author{Ri-Qi Su}
\affiliation{School of Electrical, Computer, and Energy
Engineering, Arizona State University, Tempe, AZ 85287, USA}

\author{Wen-Xu Wang}
\affiliation{Department of Systems Science, School of
Management and Center for Complexity Research,
Beijing Normal University, Beijing 100875, China}

\author{Xiao Wang}
\affiliation{School of Biological and Health Systems Engineering,
Arizona State University, Tempe, AZ 85287, USA}

\author{Ying-Cheng Lai} \email{Ying-Cheng.Lai@asu.edu}
\affiliation{School of Electrical, Computer, and Energy
Engineering, Arizona State University, Tempe, AZ 85287, USA}


\maketitle

To use the source codes, one need to have {\bf gcc} complier and {\bf Matlab}
installed on a linux/unix based operator system. The test environment is Ubuntu
{\it 14.04} with {\bf gcc} version {\it 4.9.2} and {\bf Matlab} version {\it
8.4.0.150421}, on a personal PC powered by i7-2700 CPU and equipped with 32G
memories. \\

\noindent
Before simulation, edit the Bash script {\bf simulate.sh} and specify the actual
{\bf Matlab} instalation path in Line 12, together with the predefined
parameters from Line 15 to Line 21. Within these parameters, {\bf {Stamp}}
refers to time stamp and is used to specify different simulation realizations.
{\bf Nodes} and {\bf Degree} refer to the network size and degree, respectively,
for the generated scale-free networks. {\bf DataDef} specifies the total
sampling data points collected from the generated time series. {\bf NmDef} is
the list of $R_{m}$, which is the required measurement amount normalized by the
number of total unknown parameters, and {\bf EpsDef} is the list of $\epsilon$,
or the tolerance for compressive sensing algorithm (see Method for detail
description.) Both of {\bf NmDef} and {\bf EpsDef} have to be written in the
vector format defined in {\bf Matlab}. The {\bf RepTimeDef} defines the
independent realizations in network reconstruction. \\

\noindent
Then one may open a terminal and change to the directory where the files locate, and change the
permission for {\bf simulate.sh}: \\
\indent	{\bf chmod +x ./simulate.sh} \\
\noindent Now one can execute the Bash script to perform simulation using: \\
\indent	{\bf ./simulate.sh } \\

\noindent
The script will automatically generate a geospatial scale free network and
simulate time series for the coupled dynamics on it. It will also generate the
{\bf Matlab} code to reconstruct networks and geospatial information from
simulated time series, store the reconstruction results in {\bf Matlab} data
storage format ( {\bf *.mat} file) and plot results by calling {\bf Matlab}.\\

\noindent
The required source codes for this script and their functionalities are listed as follows. \\
\begin{enumerate}
\item {\bf dataioV1.h} is a header file written by Dr. Xuan Ni (see
http://chaos34.eas.asu.edu/) and revised by Dr. Riqi Su. It provides functions
in C programming environment to read and write {\bf Matlab} data storage format,
to realize efficient data structure for matrix and vector, and to generate high
quality random variables.

\item  {\bf BANet.cpp} was used to generate specific scale free random networks of given
network sizes and average degrees.

\item  {\bf DDEv3.cpp} was used to generate time series for Rossler oscillators with delayed
coupled on given scale free networks. It will return a {\bf TimeSer*.mat} file.

\item {\bf l1-magic} codes located in directory {\bf /Optimization}. The source
codes were previously downloaded from
http://users.ece.gatech.edu/justin/l1magic/ and they are used here to solve the
sparse regulation problem. 

\item {\bf localize2d.m} was provided by A. H. Sayed et al., in Network-based
Wireless Location, {\it IEEE Signal Processing Magazine}, Vol. 24, No. 4, 2005.
It is called by {\bf triLocation.m} to solve the triangle localization problem
with given distance matrix.

\item {\bf matlab\_BANet.m} is {\bf Matlab} code to reconstruct geospatial
network from time series and was written by the authors.

\end{enumerate}

\noindent
All the simulation results presented in this paper are given as {\bf
*.mat} files. In the final stored {\bf *.mat} file, there are several importance variables.
The {\bf StoreCoup} stores $W_{nz}$, which is the avarage normalized errors for
reconstructed coupling weights of the existed links. The {\bf StoreDelay} stores
$D_{nz}$, the average normalized errors for coupling delays of the existed
links. And {\bf zStoreCoup} and {\bf zStoreDelay} represent $W_{z}$ and $D_{z}$,
respectively, which are the average of absolute error for the weights or delays
associated with non-existed links. Please refer to the Method section and Figure
3 for detailed discussion.\\
 

\end{document}
